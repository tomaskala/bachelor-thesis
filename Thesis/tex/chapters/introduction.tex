As the number of news articles published each day grows, it becomes impossible to manually examine them all to discover events that occurred in the world. The field of \textit{Event Detection} arose as a subfield of \textit{Information Retrieval} and \textit{Topic Detection and Tracking} with a goal to aid the users by automatically discovering important events in text streams.

More precisely, given a stream of text documents published over a certain time period, the task is to analyze them and output a collection of events that happened in the world during the period. An event is loosely defined as \textit{something happening in a certain place at a certain time} \cite{retrospective-online-study}.

The documents do not necessarily have to come from news streams; a lot of work has also been published in event detection by analyzing tweets, an overview can be found in \cite{twitter-survey}. The paper also distinguishes between \textit{retrospective} and \textit{online} event detection. The former analyzes a given collection of documents to discover past events, the latter (also known as \textit{First Story Detection}) tries to classify seequentially incoming documents into ``old'' documents concerning events already known, and ``new'' documents concerning events not yet seen.

Further distinction can be made between event representation. Some methods directly compare documents by their content and temporal similarity, \cite{document-bursty-representation}. Others, such as \cite{event-detection, parameter-free} and our method included, represent the events by clusters of keywords related semantically and temporarily.

We focus on unsupervised event detection such that it does not need any annotated data whatsoever, and, preferrably, does not require to fine-tune a large number of parameters. The system should also be reasonably fast, so that it can comfortably be used to browse a document collection.

We chose to modify an existing approach, \cite{event-detection}, which represents the events by clusters of related keywords. We apply the recent advances in word embeddings (see \autoref{chap:data-preprocessing}) to enrich the existing methods by a more fine-grained metric of word similarity.

{\color{red} TODO: Once we have results, mention the alternative event detection algorithm, though it appears to outperform the existing one already.}

To make the system more usable, we do not stop at the keyword-level representation, but move on to extract documents relevant to the events. We also generate short summaries to annotate the events, so the user can get an idea what the events are actually about, and whether they are worth a closer examination.