This is a list of confirmed events which occurred in 2014 that was used to evaluate Precision and Recall.

\begin{tabularx}{\linewidth}{@{}c @{}c X@{}}\toprule[1.5pt]
\bf \# & \bf Date & \bf Headline \\ \midrule
1 & June 2 & Španělský král Juan Carlos I. abdikoval a za svého nástupce určil svého syna Filipa. \\ \midrule
2 & June 4-5 & Konal se 40. summit G8 v Bruselu. \\ \midrule
3 & June 7 & Petro Porošenko složil prezidentskou přísahu a stal se prezidentem Ukrajiny. \\ \midrule
4 & June 8 & Abd al-Fattáh as-Sísí složil prezidentskou přísahu a stal se prezidentem Egypta. \\ \midrule
5 & June 10 & V izraelských prezidentských volbách byl zvolen Re'uven Rivlin. \\ \midrule
6 & June 12 & V Brazílii začalo 20. mistrovství světa ve fotbale. \\ \midrule
7 & June 15 & Andrej Kiska složil prezidentskou přísahu a stal se prezidentem Slovenska. \\ \midrule
8 & June 18 & Vůdci vojenského převratu v Turecku z roku 1980 Kenan Evren a Tahsin Şahinkaya byli odsouzeni na doživotí. \\ \midrule
9 & June 19 & Filip, asturský kníže složil přísahu a stal se králem Španělska jako Filip VI. Španělský \\ \midrule
10 & July 1 & Itálie se ujala předsednictví EU. \\ \midrule
11 & July 8 & Armáda České republiky utrpěla největší ztrátu v novodobých dějinách, kdy při sebevražedném útoku poblíž letecké základny Bagram zemřeli čtyři čeští vojáci, spolu s dalšími 12 tamními oběťmi. Pátý český voják byl těžce raněn a 14. července zemřel. \\ \midrule
12 & July 13 & Mistry světa ve fotbale se stala německá fotbalová reprezentace. \\ \midrule
13 & July 15 & Novým předsedou Evropské komise se stal lucemburský politik a bývalý premiér Jean-Claude Juncker. \\ \midrule
14 & July 17 & V oblasti bojů na východní Ukrajině se zřítil Boeing 777 malajsijských aerolinií. Zemřelo všech 295 osob na palubě. \\ \midrule
15 & July 21 & Vláda Bohuslava Sobotky vybrala nového eurokomisaře. Stane se jím ministryně pro místní rozvoj Věra Jourová, která ve výběru porazila Pavla Mertlíka. \\ \midrule
16 & August 10 & V historicky první přímé prezidentské volbě v Turecku byl zvolen premiér Recep Tayyip Erdoğan. \\ \midrule
17 & August 16-28 & Letní olympijské hry mládeže 2014 v čínském Nankingu. \\ \midrule
18 & August 19 & Americký novinář James Foley byl popraven v syrské poušti neznámým islámským radikálem, jeho smrt vyvolala v západním světe vlnu pobouření. \\ \midrule
19 & August 24 & Meziplanetární sonda New Horizons prolétla blízko L5 soustavy Slunce–Neptun. \\ \midrule
20 & August 25 & Ve sporu o amnestii Václava Klause soud schválil smír, podle něhož se bývalý hradní právník Pavel Hasenkopf na vyhlášeném znění amnestie nepodílel. \\ \midrule
21 & August 28 & Recep Tayyip Erdoğan složil prezidentskou přísahu a stal se prezidentem Turecka. \\ \midrule
22 & August 30 & Polský premiér Donald Tusk byl na summitu Evropské unie zvolen předsedou Evropské rady. \\ \midrule
23 & September 1 & Pavel Hasenkopf podal na Vratislava Mynáře trestní oznámení pro pomluvu ohledně Mynářova výroku, že Hasenkopf je jedním z autorů amnestie Václava Klause. \\ \midrule
24 & September 2 & Další americký novinář Steven Sotloff byl popraven v syrské poušti neznámým islámským radikálem, stejně jako James Foley v srpnu. \\ \midrule
25 & September 4 & Ve Vilémově se zřítil most, na kterém probíhala rekonstrukce. Zemřeli čtyři dělníci, další dva byli zraněni. \\ \midrule
26 & September 6 & Počet nakažených ebolou při celoroční epidemii se přehoupl přes 4 000. \\ \midrule
27 & September 8 & Britský následník trůnu Princ William a jeho manželka Kate oznámili, že čekají druhé dítě. \\ \midrule
28 & September 10 & Kandidátka na českou eurokomisařku Věra Jourová získala portfolio spravedlnosti, spotřebitelské politiky a rovnosti pohlaví. \\ \midrule
29 & September 13 & Islámští radikálové popravili dalšího západního zajatce, tentokrát jím byl britský humanitární pracovník David Haines. \\ \midrule
30 & September 18 & Ve Skotsku proběhlo referendum o nezávislosti na Spojeném království. Pro odtržení od Británie hlasovalo 44,7\% lidí, proti 55,3\% lidí, Skotsko tak zůstane její součástí. \\ \midrule
31 & September 20 & Náčelník Generálního štábu Armády ČR Petr Pavel byl zvolen předsedou vojenského výboru NATO. \\ \midrule
32 & September 24 & Na Pražský hrad se dostal výhružný dopis adresovaný prezidentovi Miloši Zemanovi s bílým práškem. Případ šetří policie. \\ \midrule
33 & October 3 & Prezident Miloš Zeman přijal demisi ministryně pro místní rozvoj Věry Jourové. \\ \midrule
34 & October 3 & Islámští radikálové popravili dalšího západního zajatce, stal se jím opět britský humanitární pracovník Alan Henning. \\ \midrule
35 & October 7 & Evropský parlament schválil nominaci Věry Jourové na post eurokomisařky pro spravedlnost, spotřebitelskou politiku a rovnost pohlaví. \\ \midrule
36 & October 10-11 & Proběhly volby do Senátu Parlamentu České republiky, volby do zastupitelstev obcí a volby do Zastupitelstva hlavního města Prahy. Ve volbách uspěly především vládní strany ČSSD, ANO a KDU-ČSL. \\ \midrule
37 & October 14 & Žena trpící schizofrenii pobodala na obchodní akademii ve Žďáru nad Sázavou tři studenty a zasahujícího policistu. Jeden ze studentů útok nepřežil. \\ \midrule
38 & October 16 & Ve Vrběticích došlo k výbuchu muničního skladu č. 16. Na místě zahynuli dva zaměstnanci skladu, došlo k evakuaci obyvatel přilehlých obcí. \\ \midrule
39 & October 16 & Zanikla europarlamentní frakce Evropa svobody a přímé demokracie, 20. října byla opět obnovena. \\ \midrule
40 & October 17-18 & Proběhlo druhé kolo voleb do Senátu Parlamentu České republiky. Ve volbách uspěly především vládní strany ČSSD, ANO a KDU-ČSL. \\ \midrule
41 & November 9 & V Katalánsku začalo symbolické hlasování o nezávislosti na Španělsku. \\ \midrule
42 & November 12 & Přistál modul Philae jako historicky první lidský stroj na kometě. \\ \midrule
43 & November 15 & Islámští radikálové popravili dalšího západního zajatce, stal se jím americký humanitární pracovník Peter Kassig. \\ \midrule
44 & November 15-16 & Summit G20 v Brisbane. \\ \midrule
45 & December 1 & Ledovková kalamita ochromila hromadnou dopravu v ČR a dodávky elektřiny v mnoha regionech. Tramvajová doprava v Praze dokonce poprvé ve své historii zažila úplné zastavení provozu. Do normálu se dopravní i energetická situace vrátila až 3. prosince. \\ \midrule
46 & December 1 & Druhým předsedou Evropské rady se stal Donald Tusk. \\ \midrule
47 & December 3 & Ve Vrběticích došlo k dalšímu výbuchu muničního skladu č. 12. Opět proběhla evakuace obyvatel přilehlých obcí, oba dva výbuchy jsou vyšetřovány jako úmyslný trestný čin. \\ \midrule
48 & December 16 & Ozbrojenci ze skupiny Tahrík-e Tálibán-e Pákistán spáchali masakr v péšávarské vojenské škole škole. Útok si vyžádal 141 obětí většinu z nich tvořili děti. \\ \midrule
49 & December 28 & Na cestě ze Surabaje do Singapuru se ztratilo letadlo malajsijské společnosti AirAsia se 162 lidmi na palubě. \\

\bottomrule[1.25pt]
\end{tabularx}