This is the first phase of the event detection algorithm, focused on obtaining a set of candidate words for event representation. We do not yet perform the actual event detection, which will be addressed in the next chapter, but merely extract a subset of words carrying enough information to be considered representative.

Here we work with the assumption that an event can be detected by observing the frequencies of individual words over time and grouping together those words which appear in similar documents during similar time periods \cite{event-detection, parameter-free}. This corresponds to an event being often mentioned in the text stream around the period when it actually occurred. Of course, not all words are representative of an event, so we will have to impose a criterion of a ``word eventness''.

\cite{event-detection} also distinguished between periodic and aperiodic words, where periodic words are mentioned with a certain period (these words are related for example to sport matches played every weekend, weather forecasts reported every day, etc.) They divided the words into two groups by their periodicity, and detected events from each group separately. However, during our evaluation, some word periodicities were misclassified. This would cause an event represented by those words to be split into a ``periodic part'' and an ``aperiodic part''. Therefore, we detect events from all ``eventful'' words at once, and examine the periodicities of the events later on.

{\color{red} TODO: Some example of a misclassified word}

At first, we will construct a trajectory of each word --- a measure of word frequency over time. Then, we will apply signal processing techniques which will be used to determine the eventness of each word. The same techniques will be later used to determine the event periodicities in \autoref{chap:document-retrieval}. Once we have a notion of word eventness, we will extract a small subset of words to be considered for further analysis, and discard the rest.

These word trajectories will then be examined for so called ``bursts'' in frequency, where a word would suddenly start appearing in a large number of documents during a short time period. Should a number of words appear in similar documents with overlapping bursts, it may be an indicator that an event worthy of attention occurred.

One thing to note is that the frequency of a word is, by itself, not a good indicator of a word importance.
Stopwords appearing in most documents, such as conjunctions, prepositions, etc. do not carry any information and should be ignored. Therefore, we will utilize the parts of speech tagging performed earlier and limit our analysis to Nouns, Verbs, Adjectives and Adverbs only.

{\color{red} TODO: Pretty graphs of eventful word vs. stopword trajectory}

In the following two chapters, we focus entirely on word analysis, ignoring the documents. We will return to the document representation after having assembled the words into events. The core of this algorithm is taken from \cite{event-detection}.


\section{Binary bag of words model}
To construct the word trajectories, we first need to know which words appear in which documents, as we are interested in the document frequency of each word. We will create a binary bag of words model, which is represented by a binary matrix denoting the incidence of documents and words. This model completely ignores word order, which is not necessary for this analysis.

We define a term-document matrix $\bowmat \in \left\{ 0, 1 \right\}^{\doccount \times \featcount}$, where $\doccount$ is the number of documents and $\featcount$ is the total vocabulary size. The document collection can then be interpreted as a set of $\doccount$ observations, each consisting of $\featcount$ binary features. The matrix $\bowmat$ is defined as

\begin{equation} \label{eq:bow-matrix}
	\bowmat_{ij} \coloneqq
	\begin{cases}
		1, & \text{document}~i~\text{contains the word}~j \text{;} \\
		0, & \text{otherwise.}
	\end{cases}
\end{equation}

Because every document contains only a small fraction of the vocabulary, the matrix $\bowmat$ consists mostly of zeroes. This allows us to store the matrix in a sparse format, which makes it possible to fit the matrix in memory. We use a sparse matrix instead of a more traditional inverted index \cite{information-retrieval}, because this representation allows us to vectorize some further operations.


\section{Computing word trajectories}
\cite{event-detection} defined the time trajectory of a word $w$ as a vector\\ $\vect{\traj}_{w} = \left[ \traj_{w}(1), \traj_{w}(2), \dots, \traj_{w}(\streamlen) \right]$ with each element $\traj_{w}(t)$ being the relative frequency of $w$ at time $t$. This frequency is defined using the DFIDF score:

\begin{equation}
	\traj_{w}(t) \coloneqq \underbrace{\frac{\text{\df}_{w}(t)}{\text{\doccount}(t)}}_{\text{DF}} \cdot \underbrace{\log{\frac{\doccount}{\text{\df}_{w}}}}_{\text{IDF}},
\end{equation}

where $\text{\df}_{w}(t)$ is the number of documents published on day $t$ containing the word $w$ (time-local document frequency), $\text{\doccount}(t)$ is the number of documents published on day $t$, $\doccount$ is the total number of documents and $\text{\df}_{w}$ is the number of documents containing the word $w$ (global document frequency).

These word trajectories are stored in a matrix $\trajmat \in \R^{\featcount \times \streamlen}$, with $\vect{\traj}_w$ being the $w$-th row of $\trajmat$. Here we take advantage of the normalization of the publication days, since they can now be used as column indices of $\trajmat$.

To make the computation efficient, we vectorize most of the operations. Along with the matrix $\bowmat$ defined in \ref{eq:bow-matrix}, we define a matrix $\dtdmat \in \left\{ 0, 1 \right\}^{\doccount \times \streamlen}$ mapping the documents to their publication days:

\begin{equation}
	\dtdmat_{ij} \coloneqq
	\begin{cases}
		1, & \text{document}~i~\text{was published on day}~j \text{;} \\
		0, & \text{otherwise}.
	\end{cases}
\end{equation}

Next, we sum the rows of $\bowmat$ together to obtain $\vect{\df} = \left[ \text{\df}_{1}, \text{\df}_{2}, \dots, \text{\df}_{\featcount} \right]$, and similarly the rows of $\dtdmat$ to obtain $\vect{\doccount}_{t} = \left[ \text{\doccount}(1), \text{\doccount}(2), \dots, \text{\doccount}(\streamlen) \right]$.

Using these matrices and vectors, we can compute $\trajmat$ as follows:

\begin{equation}
	\trajmat \coloneqq
		\underbrace{\text{diag} \left( \log{\frac{\doccount}{\vect{\df}}} \right)}_{\text{IDF}}
		\cdot
		\underbrace{\bowmat^{\T}
		\cdot \dtdmat
		\cdot \text{diag} \left( \frac{1}{\vect{\doccount}_{t}} \right)}_{\text{DF}}
\end{equation}

Now, having trained the word2vec model in \autoref{chap:data-preprocessing} and constructed the word trajectories, we obtained temporal and semantic representation of the words. Every word $w$ is represented by two vectors -- $\embed_{w}$ being its word2vec embedding, and $\vect{\traj}_{w}$ its time trajectory. The trajectory will be further analyzed in this chapter, while both time trajectory and word2vec embedding will be used in \autoref{chap:event-detection} to group words into events.


\section{Spectral analysis}
Having constructed the word trajectories, we still need to decide which words are eventful enough. \cite{event-detection} interpreted the word trajectories as time signals, which allowed them to analyze the trajectories using signal processing techniques. They performed the analysis both to decide word eventness and to discover the word periodicity.

Unlike the original paper, we only analyze the signal power to decide which words are eventful enough. We will detect events from both periodic and aperiodic words at once, and decide the periodicity of the whole events in \autoref{chap:document-retrieval}.

We apply the discrete Fourier transform to the trajectories to represent each time series as a linear combination of $\streamlen$ complex sinusoids. We obtain $\mathcal{F} \vect{\traj}_{w} = \left[ X_{1}, X_{2}, \dots, X_{\streamlen}\right ]$ such that

\begin{equation*}
	X_{k} = \sum_{t = 1}^{\streamlen}{\traj_{w}(t) \exp(- \frac{2 \pi \mi}{\streamlen} (k - 1) t}), ~ k = 1, 2, \dots, \streamlen.
\end{equation*}

The measure of ``eventness'' of a word is simply its signal power. That can be determined from the power spectrum of each signal, estimated using the periodogram estimator

\begin{equation*}
	\vect{P} = \left[ \|X_{1}\|^{2}, \|X_{2}\|^{2}, \dots, \|X_{\ceil{\streamlen / 2}}\|^{2} \right].
\end{equation*}

To measure the overall signal power, we define the dominant power spectrum of the word $w$ as the value of the highest peak in the power spectrum, that is

\begin{equation}
	\text{DPS}_{w} \coloneqq \max\limits_{k \leq \ceil{\streamlen / 2}}{\|X_{k}\|^{2}}.
\end{equation}

{\color{red} TODO: Plot graphs of periodic and aperiodic word showing both time trajectory and its periodogram.}

Finally, \cite{event-detection} define the set of all eventful words (EW) as those words whose trajectory signal is powerful enough. This corresponds to their occurrence in a large number of documents in a noiseless pattern:

\begin{equation}
	\text{EW} \coloneqq \left\{ w \mid \text{DPS}_{w} > \textit{DPS-bound} \right\}.
\end{equation}

where \textit{DPS-bound} can be estimated using the \textit{Heuristic stopword detection} algorithm described in \cite{event-detection}. The algorithm computes the average trajectory value and DPS values from a given seed stopwords set. The DPS boundary is then defined as the maximum DPS value of the stopwords set.