This is the first phase of the event detection algorithm, focused on obtaining a set of candidate word features for event representation. We do not yet perform the actual event detection, which will be addressed in the next chapter, but merely extract a subset of words carrying enough information to be considered representative.

Here we work with the assumption that an event can be detected by observing the frequencies of individual words over time and grouping together those words which appear in similar documents during similar time periods \cite{event-detection, parameter-free}. This corresponds to an event being often mentioned in the text stream, be it a formal news stream or a stream of social network comments, around the period when it actually occurred. Of course, not all words are representative of an event, so we will have to impose a criterion of a ``word eventness''.

\cite{event-detection} also distinguished between periodic and aperiodic words, where periodic words appear in bursts with a certain period (these words are related for example to sport matches played every weekend, weather forecasts reported every week, etc.) They divided the words into two groups by their periodicity, and detect events from each group separately. During our evaluation, some words were misclassified, which would split an event into several parts due to keywords for that event appearing in both categories. Therefore, we detect events from all ``eventful'' words at once, and examine the periodicities of the events later on.

{\color{red} TODO: Some example of a misclassified word}

At first, we will construct a trajectory of each word --- a measure of word frequency over time. Then, we will introduce signal processing techniques which will be used to determine eventness of each word by its trajectory. We will then extract eventful words and discard the rest. The same technique will later be used to determine event periodicity.

These trajectories will then be examined for so called ``bursts'' in frequency, where a word would suddenly appear in a large number of documents. Should a higher number of words appear in similar documents with overlapping bursts, it may be an indicator that an event worthy of attention occurred.

One thing to note is that the frequency of a word is, by itself, not a good indicator of a word importance.
Stopwords appearing in most documents, such as conjunctions, prepositions, etc. do not carry any information and should be ignored. Therefore, we will utilize the parts of speech tagging performed earlier and limit our analysis to Nouns, Verbs, Adjectives and Adverbs only.

{\color{red} TODO: Pretty graphs of eventful word vs. stopword trajectory}

This chapter focuses entirely on word analysis while ignoring the documents. Once we have assembled the words into events, we will return to document representation. The core of this algorithm is taken from \cite{event-detection}.


\section{Binary bag of words model}
To construct the word trajectories, we first need to know which words appear in which documents, as we are interested in the document frequency of each word. We will create a binary bag of words model, which is represented by a binary matrix denoting the incidence of documents and words. This model completely ignores word order, which is not necessary for this analysis, and was considered previously in word embeddings.

We define a term-document matrix $\bowmat \in \left\{ 0, 1 \right\}^{\doccount \times \featcount}$, where $\doccount$ is the number of documents and $\featcount$ is the total vocabulary size. The document collection can then be interpreted as a set of $\doccount$ observations, each consisting of $\featcount$ binary features. The matrix $\bowmat$ is defined as

\begin{equation} \label{eq:bow-matrix}
	\bowmat_{ij} \coloneqq
	\begin{cases}
		1, & \text{document}~i~\text{contains the word}~j \text{;} \\
		0, & \text{otherwise.}
	\end{cases}
\end{equation}

To further limit the feature space, we discard words appearing in less than 30 documents or in more than 90\% of the documents. The idea behind this is that the words appearing only in few documents cannot possibly represent relevant events, and are mostly anomalies or typos. On the other hand, words appearing in most of the documents are likely stopwords, and do not carry much information. This helps to prune the feature space and makes $\bowmat$ reasonably sized.


\section{Computing feature trajectories}
The time trajectory of a word feature $f$ is a vector $\vect{\traj}_{f} = \left[ \traj_{f}(1), \traj_{f}(2), \dots, \traj_{f}(\streamlen) \right]$ with each element $\traj_{f}(t)$ being the relative frequency of $f$ at time $t$. This frequency is defined using the DFIDF score:

\begin{equation}
	\traj_{f}(t) \coloneqq \underbrace{\frac{\text{\df}_{f}(t)}{\text{\doccount}(t)}}_{\text{DF}} \cdot \underbrace{\log{\frac{\doccount}{\text{\df}_{f}}}}_{\text{IDF}},
\end{equation}

where $\text{\df}_{f}(t)$ is the number of documents published on day $t$ containing the feature $f$ (time-local document frequency), $\text{\doccount}(t)$ is the number of documents published on day $t$ and $\text{\df}_{f}$ is the number of documents containing the feature $f$ (global document frequency).

These feature trajectories are stored in a matrix $\trajmat \in \R^{\featcount \times \streamlen}$, with $\vect{\traj}_f$ being the $f$-th row of $\trajmat$. Here we take advantage of the normalization of the publication days, since they can now be used as column indices of $\trajmat$.

To make the computation efficient, we vectorize most of the operations. Along with the matrix $\bowmat$ defined in \ref{eq:bow-matrix}, we define a matrix $\dtdmat \in \left\{ 0, 1 \right\}^{\doccount \times \streamlen}$ mapping the documents to their publication days:

\begin{equation}
	\dtdmat_{ij} \coloneqq
	\begin{cases}
		1, & \text{document}~i~\text{was published on day}~j \text{;} \\
		0, & \text{otherwise}.
	\end{cases}
\end{equation}

Next, we sum the rows of $\bowmat$ together to obtain $\vect{\df} = \left[ \text{\df}_{1}, \text{\df}_{2}, \dots, \text{\df}_{\featcount} \right]$, and similarly the rows of $\dtdmat$ to obtain $\vect{\doccount}_{t} = \left[ \text{\doccount}(1), \text{\doccount}(2), \dots, \text{\doccount}(\streamlen) \right]$.

Using these matrices and vectors, we can compute $\trajmat$ as follows:

\begin{equation}
	\trajmat \coloneqq
		\underbrace{\text{diag} \left( \log{\frac{\doccount}{\vect{\df}}} \right)}_{\text{IDF}}
		\cdot
		\underbrace{\bowmat^{\T}
		\cdot \dtdmat
		\cdot \text{diag} \left( \frac{1}{\vect{\doccount}_{t}} \right)}_{\text{DF}}
\end{equation}

Now, every word feature $f$ is represented by two vectors --- $\vect{\traj}_{f}$ being its time trajectory, and $\embed_{f}$ being its word embedding. We obtained temporal and semantic representation of the words, both of which will be utilized for event detection in the next chapter.


\section{Spectral analysis}
Having constructed the word trajectories, we still need to decide which words are eventful enough. We will interpret the word trajectories as signals, which allows us to analyze them using signal processing techniques. Here, we will analyze only the signal power to measure the eventness, but the same technique will be used in \autoref{chap:document-retrieval} to discover event periodicity.

We apply the discrete Fourier transform to each trajectory to represents the time series as a linear combination of $\streamlen$ complex sinusoids. We obtain $\mathcal{F} \vect{\traj}_{f} = \left[ X_{1}, X_{2}, \dots, X_{\streamlen}\right ]$ such that

\begin{equation*}
	X_{k} = \sum_{t = 1}^{\streamlen}{\traj_{f}(t) \exp(- \frac{2 \pi \mi}{\streamlen} (k - 1) t}), ~ k = 1, 2, \dots, \streamlen.
\end{equation*}

Having moved from the time domain to the frequency domain, we can now analyze the signal power of each word. They will be retrieved from the power spectra of the transformed trajectories, which are estimated using the periodogram estimator

\begin{equation*}
	\vect{P} = \left[ \|X_{1}\|^{2}, \|X_{2}\|^{2}, \dots, \|X_{\ceil{\streamlen / 2}}\|^{2} \right].
\end{equation*}

To measure the overall signal power, we define the dominant power spectrum of the word feature $f$ as the value of the highest peak in the power spectrum, that is

\begin{equation}
	\text{DPS}_{f} \coloneqq \max\limits_{k \leq \ceil{\streamlen / 2}}{\|X_{k}\|^{2}}.
\end{equation}

{\color{red} TODO: Plot graphs of periodic and aperiodic word showing both time trajectory and its Fourier transform}

When applied to rows of the matrix $\trajmat$, this method yields a vector $\vect{DPS} \in \R^{\featcount}$ containing the dominant power spectra.

We finally define the set of all eventful words as those whose trajectory signal is powerful enough, that is, they appear in a large number of documents in a noiseless pattern:

\begin{equation}
	\text{EW} \coloneqq \left\{ f \mid \text{DPS}_{f} > \textit{dps-bound} \right\}.
\end{equation}

{\color{red}TODO: Define dps-bound!}