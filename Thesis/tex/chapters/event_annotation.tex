The final step of our method is to annotate the detected events in a human-readable way. We aim to generate short summaries so that the user does not have to process a large quantity of text, and can just skim through a few sentences to decide whether they are interested in that particular event. If so, then they can examine the event more closely and go through the actual documents, which we have retrieved in chapter \autoref{chap:document-retrieval}.

Although the keyword set discovered in \autoref{chap:event-detection} provides a concise representation of an event, it can lead to ambiguities or simply not reveal enough information. {\color{red} TODO: Add an example of an event not well recognizable by its keyword set.} The keywords should be considered an internal representation used in the detection process, not something presentable to the user.

A simple method is to annotate an event by the headline of the most relevant document in terms of Word Mover's Similarity. This may give insight of the general topic of the particular event, but it is unlikely that a whole event will be well characterized by a single document. For this reason, we also investigate a more complex method. Nevertheless, we will provide such headlines for comparison.

The second approach is to use multi-document summarization techniques to generate a short summary of an event's document set. More specifically, we attempt to extract some number of sentences from the document set which cover the general topic of the set without providing redundant information. As the documents come from different sources and describe the events from different perspectives, the result will not generally be a continuous paragraph, but more of a set of characteristic sentences. Still, a longer piece of text will likely provide a better insight into an event than a single headline.

We use the multi-document summarization system presented in \cite{multi-summarization-1, multi-summarization-2}, which we describe in more detail in the following paragraphs. This system was later improved by \cite{mogren-1}, who evaluated the usage of different word embedding techniques for sentence similarity measures. Their work led to the system presented in \cite{mogren-2} which aggregates several different similarity measures to obtain a better quality summary.


\section{Multi-document summarization}
In \cite{multi-summarization-1}, the authors formulate the task of multi-document summarization as a constrained combinatorial optimization problem, where the goal is to retrieve a subset of sentences maximizing a monotone submodular function $\quality{\cdot}$ measuring the summary quality.

A submodular function $\quality{\cdot}$ on a set of sentences $V$ satisfies the property of \textit{diminishing returns}; that is, for $A \subseteq B \subseteq V \setminus \{ v \},\ \quality{A \cup \{ v \}} - \quality{A} \geq \quality{B \cup \{ v \}} - \quality{B},\ v \in V$. This has an intuitive interpretation for text summarization, namely that adding a sentence $v$ to a longer summary does not improve the summary as much as adding it to a smaller one. The reason is that the information carried by $v$ is more likely present in the longer summary already.

Even though solving the task exactly is NP-hard, a greedy algorithm is guaranteed to find a solution inly a constant factor off the optimum, as discussed by the authors.

The summary quality is measured in terms of how representative it is to the whole set (coverage) and how dissimilar the sentences are to each other (diversity). The constraints limit the summary to a reasonable length by bounding either the number of sentences or the number of words.

In \cite{multi-summarization-1}, basic submodular functions to be used in multi-document summarization are described. In \cite{multi-summarization-2}, these functions are further developed to better capture the semantic properties.

Mathematically, the task is formulated as

\begin{equation}
\begin{alignedat}{-1}
\max_{S \subseteq V} & \quad \quality{S} = \coverage{S} + \lambda \diversity{S} \\
\text{s. t.} & \quad \sum_{i \in S}{\sentcost_{i}} \leq \budget,
\end{alignedat}
\end{equation}

where $V$ is the set of all sentences from the document set being summarized, $\sentcost_{i}$ is the cost of sentence $i$ (either 1 when limiting the number of sentences or the number of words in the sentence) and $\budget$ is the total budget.

A feasible set $S$ maximizing $\quality{\cdot}$ will provide a reasonable number of sentences well capturing the overall topic of the whole document set, no two of which being redundant. What remains is to define the coverage function $\coverage{\cdot}$ and diversity function $\diversity{\cdot}$, whose influence can be controlled by the parameter $\lambda \geq 0$. Additionally, the functions must be defined in a way that the conditions from \cite{multi-summarization-1} are not broken, so that a greedy algorithm can still be used.


\section{Coverage function}

In \cite{multi-summarization-2}, the coverage function $\coverage{\cdot}$ is defined in terms of pairwise sentence similarity $\semsim{\cdot}{\cdot}$ as

\begin{equation}
\coverage{S} = \sum_{i \in V}{\min{\Big\{ \sum_{j \in S}{\semsim{i}{j}}, \alpha \sum_{k \in V}{\semsim{i}{k}} \Big\} }}.
\end{equation}

The first argument measures the similarity between the sentence $i$ and the summary $S$, while the second argument measures the similarity between the sentence $i$ and the rest of the sentences $V$. The number $\alpha \in [0,1]$ is a threshold coefficient.

The authors further prove that if $\semsim{i}{j} \in [0,1]\ \forall i, j \in V$, the whole function remains submodular. {\color{red} TODO: Where was that?}

Originally, only a simple cosine similarity between TFIDF sentence vectors \cite{information-retrieval} was used in $\semsim{\cdot}{\cdot}$. Kågebäck et al. examined various methods of word embeddings in \cite{mogren-1} to obtain a finer measure of similarity. This alone outperformed the original method, and in \cite{mogren-2}, a more complex system aggregating multiple similarity measures was built. We use this method with several different similarity measures fit for the event detection task.

The sentence similarity $\semsim{i}{j}$ will be computed as a product of these individual similarities, all bounded in $[0, 1]$: $\semsim{i}{j} = \prod_{l}{\similarity_{s_{i}s_{j}}^{l}}$.

Next, we describe the individual sentence similarities used.

\subsection{TFIDF similarity}

The first measure used is the standard cosine similarity between TFIDF vectors \cite{information-retrieval} of two sentences $s_{i}$ and $s_{j}$. Such method is a simple measure of document similarity often used in information retrieval. As noted by \cite{multi-summarization-2}, this method often performs well enough to be used by itself.

If we denote the frequency of the word $w$ in sentence $s_{i}$ as $\text{tf}_{w,i}$ and the inverse document frequency of $w$ as $\text{idf}_{w}$, the similarity is written as

\begin{equation}
	\similarity_{s_{i}s_{j}} = \frac{\sum_{w \in s_{i} \cup s_{j}} \text{tf}_{w,i} \times \text{tf}_{w,j} \times \text{idf}_{w}^{2}}{\sqrt{\sum_{w \in s_{i}} \text{tf}_{w, i}^{2} \times \text{idf}_{w}^{2}} \sqrt{\sum_{w \in s_{j}} \text{tf}_{w, j}^{2} \times \text{idf}_{w}^{2}}}.
\end{equation}

In this case, the term frequencies are always non-negative, and so the whole cosine similarity is in $[0, 1]$.

The major setback of TFIDF similarity is that it does not go beyond simple word overlap, though weighted in a clever way. That means that should two sentences convey essentially the same information through different words, they would not be ranked similar. That can be a problem in larger document collections from different sources, with each document using a slightly different vocabulary.

\subsection{Word2Vec similarity}

We attempt to solve this problem by considering the word embeddings of the individual sentence words, as first examined by \cite{mogren-1}.

We represent a sentence $s_{i}$ by summing together the vector embeddings of its words, $\embed_{i} = \sum_{w \in s_{i}} \embed_{w}$. The similarity of two sentences is then the cosine similarity of these vectors, transformed to $[0, 1]$:

\begin{equation}
	\similarity_{s_{i}, s_{j}} = \left(\frac{\inp{\embed_{i}}{\embed_{j}}}{\| \embed_{i} \| \| \embed_{j} \|} + 1 \right) / \ 2.
\end{equation}

This similarity brings a finer distinction of word-level semantics and can handle for example synonyms well. This means that even if two sources reporting the same event use fairly different vocabularies, the sentences will still be ranked similar.

\subsection{LSI similarity}

The next similarity uses Latent Semantic Indexing \cite{lsi} to project the sentence TFIDF vectors to a latent space, whose dimensions can be interpreted as topics of the document collection. We measure the cosine similarity between sentence vectors in this space to compare the sentence topics.

Some level of topical similarity was obtained previously from Word2Vec similarity. That measure, however, compared only the individual words based on their semantics. In this measure, we attempt to concern the topic among whole sentences with respect to the whole sentence set.

The similarity is once again defined as the cosine similarity of sentence vectors from the latent space, transformed into $[0, 1]$.

\subsection{Keyword similarity}

The final similarity comes from \cite{mogren-2} and measures the degree of overlap of the sentences with a predefined keyword set. Namely, we use the keyword representation of the events obtained previously. The similarity is defined as 

\begin{equation}
	\similarity_{s_{i}, s_{j}} = \frac{\sum_{w \in \left( s_{i} \cap s_{j} \cap \text{KW} \right)} \text{tf}_{w} \text{idf}_{w}}{\left| s_{i} \right| + \left| s_{j} \right|}.
\end{equation}

The measure effectively chooses only those sentences having non-zero word overlap with the keyword set. This requirement breaks the continuity of the text, making the summary more of a set of sentences characteristic for the event.

This does not necessarily mean losing some degree of summary quality. Since the sentences come from a potentially large set of different documents from different authors, the chance of appending two sentences in a way that they make sense is quite small. On the other hand, requiring that they both share some words with the keywords characteristic for the event will severely limit the candidate sentences, choosing only relevant ones. 


\section{Diversity function}