\thispagestyle{plain}

\begin{center}
	\Large
	\textbf{Abstract}
\end{center}

Event detection is a process of analysis of text documents aiming to uncover real events happening in the world. It is based on the assumption that words appearing in similar documents and time windows are likely to concern the same real-world event. Therefore, our method attempts to group together words with similar temporal and semantic characteristics while discarding noisy words, not contributing to anything of interest. This results in a concise event representation through a set of representative keywords. These are then used to query the document collection to retrieve the actual event-related documents. Finally, we extract short summaries from these documents and annotate the events in a human-readable fashion. The keyword retrieval phase of our method is based on an existing event detection system, which we modify by employing a word embedding model to measure semantic similarity. The system is evaluated on a collection of 2 million documents from Czech news over a 13 months period and compared to the original method, not depending on word embeddings.
\\
\\
\textbf{Keywords:} Document retrieval, event detection, multi-document summarization, word embedding.

\hfill

\begin{center}
	\Large
	\textbf{Abstrakt}
\end{center}

Detekce událostí je proces analýzy textových dokumentů za účelem odhalení událostí, které se během doby jejich vydání staly ve světě. Tento proces je založen na předpokladu, že sémanticky podobná slova se zvýšeným výskytem během stejného období se pravděpodobně vztahují ke stejné události. Námi zkoumaná metoda se tedy snaží shlukovat dohromady slova s podobnou časovou nebo sémantickou charakteristikou, a zároveň ignorovat slova nenesoucí žádnou informaci. To vede k jednoduché reprezentaci událostí pomocí skupin klíčových slov. Tato klíčová slova jsou následně použita k dotazu do zkoumané kolekce a získání dokumentů vztahujících se k jednotlivým událostem. Z těchto dokumentů jsou nakonec extrahována krátká shrnutí pro bohatší popis událostí. Fáze získávání klíčových slov je založena na existujícím postupu, který modifikujeme použitím modelu vnořování slov (word embedding) k měření sémantické podobnosti. Metoda je vyhodnocena na kolekci 2 milionů dokumentů z českých novinových serverů vydané za období 13 měsíců, a porovnána s původním postupem nevyžadujícím vnořování slov.
\\
\\
\textbf{Klíčová slova:} Získávání dokumentů, detekce událostí, sumarizace více dokumentů, word embedding.