\thispagestyle{plain}

\begin{center}
	\Large
	\textbf{Abstract}
\end{center}

We present an unsupervised framework for retrospective event detection from text streams. The process of event detection is based on the assumption that words appearing in similar documents and time windows are likely to concern the same real-world event. Therefore, the method attemps to group together words with similar temporal or semantic characteristics while discarding noisy words, not contributing to anything of interest. This results in a simple event representation through a set of keywords. These are then used to query the document collection to obtain the actual event-related documents. Finally, we extract short summaries from these documents to annotate the events in a human-readable fashion. The keyword retrieval phase of our method is based on an existing event detection system, which we modify by employing a word embedding model to measure semantic similarity. The framework is evaluated on a collection of 2 million documents from Czech news over a 13 months period and compared to the original method, not depending on word embeddings.
\\
\\
\textbf{Keywords:} Document retrieval, event detection, multi-document summarization, word embedding.