\thispagestyle{plain}

\begin{center}
	\Large
	\textbf{Abstract}
\end{center}

We present a framework for retrospective event detection from text streams. The method works in a completely unsupervised manner, and does not require to fine-tune a large number of parameters. The process of event detection is based on the assumption that words appearing in similar documents in the same time window are likely to concern the same real-world event, which has been highly reported in the text stream at that time. Therefore, the method attempts to discover such event keywords by filtering out words with noisy temporal or semantic characteristic, unlikely to be related to anything of interest. These keywords are further used to query the document collection to obtain the actual event-related documents, which are finally used to generate human-readable annotations. The keyword detection phase of our method is based on an existing event detection system, which we enrich by using word embeddings to measure semantic similarity. The method is evaluated on a collection of 2 million documents from Czech news over a 13 months period, and compared to the original method, not depending on word embeddings.