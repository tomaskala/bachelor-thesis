\thispagestyle{plain}

\begin{center}
	\Large
	\textbf{Abstract}
\end{center}

We present an unsupervised framework for retrospective event detection from text streams. The process of event detection is based on the assumption that words appearing in similar documents and time windows are likely to concern the same real-world event. Therefore, the method attemps to group together words with similar temporal or semantic characteristics while discarding noisy words, not contributing to anything of interest. This results in a simple event representation through a set of keywords. These are then used to query the document collection to obtain the actual event-related documents. Finally, we extract short summaries from these documents to annotate the events in a human-readable fashion. The keyword retrieval phase of our method is based on an existing event detection system, which we modify by employing a word embedding model to measure semantic similarity. The framework is evaluated on a collection of 2 million documents from Czech news over a 13 months period and compared to the original method, not depending on word embeddings.
\\
\\
\textbf{Keywords:} Document retrieval, event detection, multi-document summarization, word embedding.

\hfill

\begin{center}
	\Large
	\textbf{Abstrakt}
\end{center}


Popíšeme systém pro retrospektivní detekci událostí z kolekcí textových dokumentů fungující bez učitele. Proces detekce událostí je založen na předpokladu, že slova vyskytující se v podobných dokumentech a časových intervalech se pravděpodobně vztahují ke stejné události, která se udála ve světě. Metoda se tedy snaží shlukovat dohromady slova s podobnou časovou nebo sémantickou charakteristikou a zároveň ignorovat slova nenesoucí žádnou informaci. To vede k jednoduché reprezentaci událostí pomocí skupin klíčových slov. Tato klíčová slova jsou následně použita k dotazu do zkoumané kolekce a získání dokumentů vztahujících se k jednotlivým událostem. Z těchto dokumentů nakonec extrahujeme krátká shrnutí, pomocí kterých popíšeme události lidsky čitelným způsobem. Fáze získávání klíčových slov je založena na existujícím postupu který modifikujeme použitím modelu vnořování slov (word embedding) k měření sémantické podobnosti. Metoda je vyhodnocena na kolekci 2 milionů dokumentů z českých novinových serverů vydané za období 13 měsíců, a porovnána s původním postupem nevyžadujícím vnořování slov.
\\
\\
\textbf{Klíčová slova:} Získávání dokumentů, detekce událostí, sumarizace více dokumentů, word embedding.